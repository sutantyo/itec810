\documentclass{article}
\usepackage[T1]{fontenc}
\usepackage[inline]{asymptote}
\usepackage{pslatex}
\usepackage{color}  
\usepackage{graphicx}     
\usepackage{verbatim}
\usepackage{xcolor}
\usepackage{paralist}
\usepackage{tagging}

\usepackage[colorlinks=true,urlcolor=red]{hyperref}
\setlength{\topmargin}{-0.5in}                  % topmargin now at 1in
\setlength{\textheight}{9.5in}                  % body of text = 9.5in
\setlength{\oddsidemargin}{0in}                 % left margin = 1.0in on odd-numbered pages
\setlength{\evensidemargin}{0in}                % left margin = 1.0in on even-numbered pages 
\setlength{\textwidth}{6.5in}                   % width of text line.
\setlength{\parindent}{0.0in}
\newcommand{\code}{\texttt}
\usepackage{listings}
\lstset{%
	language=Java,
	basicstyle=\footnotesize\ttfamily,
	numbers=left,
	numberstyle=\tiny,        
	xleftmargin=17pt,
        	xrightmargin=5pt,
	frame=single,
	breaklines=true,
	moredelim=**[is][\color{red}]{@}{@}
}

\lstdefinestyle{buggy}{
  language=Java,
  emptylines=1,
  breaklines=true,
  basicstyle=\ttfamily\color{black},
  moredelim=**[is][\color{red}]{@}{@},
}

\lstdefinestyle{correct}{
  language=Java,
  emptylines=1,
  breaklines=true,
  basicstyle=\ttfamily\color{black},
  moredelim=**[is][\color{blue}]{@}{@},
}

\usepackage{listings}
\lstset{%
	language=Java,
	basicstyle=\footnotesize\ttfamily,
	numbers=left,
	numberstyle=\tiny,        
	xleftmargin=17pt,
        	xrightmargin=5pt,
	frame=single,
	breaklines=true,
	moredelim=**[is][\color{red}]{@}{@}
}

\begin{document}

\definecolor{aquamarine}{rgb}{0,0,0.7}
\definecolor{blue}{rgb}{0,0,0.7}
\definecolor{red}{rgb}{1,0,0}


%\renewcommand{\labelenumi}{\arabic{enumi}.}
\renewcommand{\labelenumi}{\alph{enumi}.}

\title{Template documentation}

This document describes the components for generating a question template in the randomized quiz generator.

\begin{enumerate}
\item Any statement of the code that should NOT be displayed can be hidden by putting a \texttt{//HIDE} comment at the end of the statement.
\item General structure of question is as follows where the questions is \emph{What is the value of result} when the following code is executed,

\begin{lstlisting}[style=buggy]
class ExampleProgram { //HIDE
    public static void main(String[] args) { //HIDE
   @ code visible to the user @
    System.out.println(result); //HIDE
}}//HIDE
	
\end{lstlisting}

\item $rand(low, high)$: gives a value in the range [low, high]
\item $randset(array(item1, item2...))$ gives a random item from the set
\item An existing variable can be accessed is subsequent substitutions by surrounding it with \%. For example, $s1$ can be accessed by including $\%s1\%$.
\item Selection from existing variables can be done in the following way, in which $s5$ is either $s4 - 1$, or $s4$ or $s4 + 1$ with equal probablities. One can manipulate the probablities, and therefore distribution, using appropriate conditions.
\begin{lstlisting}
//Equal chance of any condition happenning
$decidor=rand(0,2);
if($decidor==0) {
    return %s4% + 1; //First result
}
elseif($decidor==1) {
    return %s4%;
}
else {
    return %s4% - 1;
}
\end{lstlisting}

\item \texttt{randset} can be used to randomize variables AND operators. Some examples,
\begin{itemize}
\item $s1 = randset(array(``true'', ``false''), s3 = randset(array("!!","!","!!!")), val = `s3``s1`$.
\item $s1 = rand(1, 10), s2 = rand(1, 5)$, \texttt{double val = `s1`.`s2`;}
\end{itemize}
\end{enumerate}


\end{document}
